\documentclass[a4paper]{report}

% Packages
\usepackage{listings}
\usepackage{color}
\usepackage[utf8]{inputenc}
\usepackage{listingsutf8}
\usepackage{graphicx}
\usepackage{epstopdf}
\usepackage{fancyhdr}
\usepackage[T1]{fontenc}
\usepackage{pmboxdraw}
\usepackage{parskip}
\usepackage[top=2cm, bottom=2cm, left=2cm, right=2cm]{geometry} % Les marges.

\definecolor{mygreen}{rgb}{0,0.6,0}
\definecolor{mygray}{rgb}{0.5,0.5,0.5}
\definecolor{mymauve}{rgb}{0.58,0,0.82}
\definecolor{bggray}{rgb}{0.95, 0.95, 0.95}
\lstset{inputencoding=utf8/latin1}
\lstset{ %
    backgroundcolor=\color{bggray},   % choose the background color; you must add \usepackage{color} or \usepackage{xcolor}
    basicstyle=\footnotesize,        % the size of the fonts that are used for the code
    breakatwhitespace=false,         % sets if automatic breaks should only happen at whitespace
    breaklines=true,                 % sets automatic line breaking
    captionpos=b,                    % sets the caption-position to bottom
    commentstyle=\color{mygreen},    % comment style
    deletekeywords={...},            % if you want to delete keywords from the given language
    escapeinside={\%*}{*)},          % if you want to add LaTeX within your code
    extendedchars=true,              % lets you use non-ASCII characters; for 8-bits encodings only, does not work with UTF-8
    frame=single,                    % adds a frame around the code
    frameround=tttt                  % tttt for having the corner round.
    keepspaces=true,                 % keeps spaces in text, useful for keeping indentation of code (possibly needs columns=flexible)
    keywordstyle=\color{blue},       % keyword style
    language=Matlab,                 % the language of the code
    morekeywords={*,...},            % if you want to add more keywords to the set
    numbers=left,                    % where to put the line-numbers; possible values are (none, left, right)
    numbersep=5pt,                   % how far the line-numbers are from the code
    numberstyle=\tiny\color{mygray}, % the style that is used for the line-numbers
    rulecolor=\color{black},         % if not set, the frame-color may be changed on line-breaks within not-black text (e.g. comments (green here))
    showspaces=false,                % show spaces everywhere adding particular underscores; it overrides 'showstringspaces'
    showstringspaces=false,          % underline spaces within strings only
    showtabs=false,                  % show tabs within strings adding particular underscores
    stepnumber=1,                    % the step between two line-numbers. If it's 1, each line will be numbered
    stringstyle=\color{mymauve},     % string literal style
    tabsize=2,                       % sets default tabsize to 2 spaces
    title=\lstname                   % show the filename of files included with \lstinputlisting; also try caption instead of title
}

% Header
\pagestyle{fancy}
\fancyhead[L]{Rudolf Höhn}
\fancyhead[R]{\today}


\title{TP2 - Friends Account\\Développements et services web}
\author{Rudolf Höhn}
\date{\today}


\begin{document}
\maketitle

\tableofcontents

\chapter{Aspects techniques}
\section{Architecture du projet}
\subsection{Technologies}
Pour ce projet, j'utilise \textbf{Node JS} pour le back-end et \textbf{Angular JS} pour le front-end.
L'architecture du projet est donc différente d'un site web PHP malgré qu'on retrouve quelques points semblables.\\\\
La communication entre le back-end et le front-end se fait à travers des services REST. Un document annexe explique toute l'API.
\subsection{Arborescence du projet}
Toute le premiere degré de l'arbre ci-dessous est utilisé par Node JS et toute la partie Angular est à l'intérieur du dossier \textit{public}.
Pour le front-end, j'ai décidé d'organiser les fichiers par besoin et non par type. Certains développeur préconisent de séparer les controlleurs, des services et des vues.
D'autres préférent rassembler les fichiers par tâche. Par exemple, dans le dossier \textit{event}, nous allons retrouver tous les fichiers utilisés pour la gestion des événements (qui est la partie principale de l'application).
\begin{verbatim}
    .
    ├── README.md
    ├── app
    │   └── models
    │       ├── event.js
    │       └── user.js
    ├── config.js
    ├── database_model.json
    ├── db/
    ├── doc/
    ├── package.json
    ├── public
    │   ├── app-content/*.css
    │   ├── app-services
    │   │   ├── authentication.service.js
    │   │   ├── event.service.js
    │   │   └── user.service.js
    │   ├── app.js
    │   ├── event
    │   │   ├── event.controller.js
    │   │   ├── event.view.html
    │   │   ├── modal.controller.js
    │   │   ├── modalAddEvent.html
    │   │   ├── modalAddSpending.html
    │   │   ├── modalMembers.html
    │   │   └── modalUpSpending.html
    │   ├── img/
    │   ├── index.html
    │   ├── lib/
    │   ├── login
    │   │   ├── login.controller.js
    │   │   └── login.view.html
    │   └── register
    │       ├── register.controller.js
    │       └── register.view.html
    └── server.js
\end{verbatim}

\section{Base de données}
\subsection{Technologie}
Dans le cadre de ce TP, je me suis orienté vers une base NoSql.
En me documentant sur le web, j'ai vu que beaucoup de développeurs utilisaient le \textit{MEAN Stack} (MongoDB, ExpressJS, AngularJS, NodeJS) pour leurs applications web.
J'ai donc décidé de suivre cette architecture et donc de prendre une base MongoDB.

\subsection{Modèle}
Comme MongoDB est une base NoSql, le modèle n'a pas lieu d'être en relationnel, c'est pour cela qu'il est présenté sous la forme JSON.
\begin{verbatim}
"Client" : {
    "_id":"ID",
    "name":"String",
    "password":"String"
}
"Event" :
{
    "_id": "ID",
    "label": "String",
    "currency": "String",
    "spending": [
      {
        "_id": "ID",
        "label": "String",
        "amount": "Number",
        "author": "String",
        "concerned": [
          {
            "_id": "ID",
            "name": "String",
            "weight": "Number",
          }
        ],
        "date": "Date"
      }
    ],
    "clients": [
      {
        "_id": "ID",
        "name": "String",
        "weight": "Number"
      }
    ]
  }
\end{verbatim}

\section{Fonctionnalités}
Toutes les fonctionnlités listées ci-dessous sont utilisables malgré certains bugs qui seront détaillés plus bas dans le document.
\subsection{Authentification}
Pour l'authentification, j'ai utilisé les JWT (JSON Web Token). Ce sont des tokens qui sont envoyés à chaque requête vers l'API REST pour authentifier le client.
Certaines routes comme l'authentification elle-même ou la création de compte ne nécessitent pas de JWT.\\\\
Les fonctionnalités suivantes sont fonctionnelles.
\begin{itemize}
\item Création d'un compte
\item Authentification d'un utilisateur
\end{itemize}

\subsection{Gestion des événements}
Dans la gestion des évévements, les différentes fonctionnalités ci-dessous sont disponiblies.
\begin{itemize}
\item Ajouter un événement
\item Mettre à jour les membres de l'événement (personne et poids dans l'événement)
\item Modifier le titre ou la devise de l'événement
\item Supprimer un événement
\end{itemize}

\subsection{Gestion des dépenses}
\begin{itemize}
\item Ajouter une dépense
\item Modifier une dépense (coût, nom, date, personnes concernées)
\item Supprimer une dépense
\end{itemize}

\subsection{Balance}
\begin{itemize}
\item Savoir ce que chaque membre d'un événement à payer
\item Recevoir une proposition de réglement des comptes.
\end{itemize}

\section{Bugs}
Voici la liste des différents bugs non résolus.
\begin{itemize}
\item Les fenêtres \textit{Modal} ne se ferment pas toutes seules
\item Les JWT ne sont pas correctement enregistrés dans les cookies et donc, si on rafraîchit la page, on doit se reconnecter
\item Peu de champs sont vérifiés et donc, beaucoup de failles sont encore existantes.
\end{itemize}

\chapter{Installation}
\section{Procédure}
Toute la procédure ci-dessous est expliquée dans un fichier README dans le projet.\\\\
Pour faire tourner l'application en local, c'est très simple. Il faut par contre avoir d'installer sur sa machine :
\begin{itemize}
\item npm
\item node
\item mongod
\end{itemize}
Premièrement, on doit avoir une instance MongoDB qui tourne sur notre machine. MongoDB a besoin d'un dossier 'db' pour s'exécuter.
\begin{verbatim}
mongod --dbpath /path/to/db
\end{verbatim}
Maintenant que notre base de données est lancée, il nous reste plus qu'à installer les paquets requis et lancer le serveur.
\begin{verbatim}
npm install
node server.js
\end{verbatim}
Vous pouvez maintenant aller sur la page \url{http://localhost:8080} et tester l'application !\\\\
\section{Données de test}
Pour une utilisation plus rapide, vous pouvez interroger la route \textit{/setup} de l'API est charger des données de test dans la base.\\\\
Les utilisateurs inscrits dans la base :
\begin{itemize}
\item Username : \textbf{lemur} - Password : \textbf{11}
\item Username : \textbf{azara} - Password : \textbf{11}
\end{itemize}

\chapter{Manuel d'utilisateur}
Toute l'application web est sur une seule page.\\\\
La barre latérale de gauche permet de parcourir les événements liés à votre compte.\\\\
La partie centrale permet de modifier les détails de l'événement que vous parcourez ainsi que ses membres.
De plus, c'est à cet endroit que vous retrouverez la liste des dépenses du groupe. Vous pouvez en ajouter ou bien modifier un déjà existant.\\\\
La partie latérale de droite permet d'avoir le bilan financier de l'événement ainsi qu'une proposition de remboursement entre les membres de l'événement.


\end{document}
